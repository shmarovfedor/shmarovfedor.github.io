%%%%%%%%%%%%%%%%%%%%%%%%%%%%%%%%%%%%%%%%%
% Medium Length Professional CV
% LaTeX Template
% Version 2.0 (8/5/13)
%
% This template has been downloaded from:
% http://www.LaTeXTemplates.com
%
% Original author:
% Trey Hunner (http://www.treyhunner.com/)
%
% Important note:
% This template requires the resume.cls file to be in the same directory as the
% .tex file. The resume.cls file provides the resume style used for structuring the
% document.
%
%%%%%%%%%%%%%%%%%%%%%%%%%%%%%%%%%%%%%%%%%

%----------------------------------------------------------------------------------------
%	PACKAGES AND OTHER DOCUMENT CONFIGURATIONS
%----------------------------------------------------------------------------------------

\documentclass{resume} % Use the custom resume.cls style

% \makeatletter
% \newlength{\bibhang}
% \setlength{\bibhang}{1em} %1em}
% \newlength{\bibsep}
%  {\@listi \global\bibsep\itemsep \global\advance\bibsep by\parsep}
% \newenvironment{bibsection}%
%         {\begin{enumerate}{}{%
% %        {\begin{list}{}{%
%        \setlength{\leftmargin}{\bibhang}%
%        \setlength{\itemindent}{-\leftmargin}%
%        \setlength{\itemsep}{\bibsep}%
%        \setlength{\parsep}{\z@}%
%         \setlength{\partopsep}{0pt}%
%         \setlength{\topsep}{0pt}}}
%         {\end{enumerate}\vspace{-.6\baselineskip}}
% %        {\end{list}\vspace{-.6\baselineskip}}
% \makeatother

\usepackage[left=0.75in,top=0.6in,right=0.75in,bottom=0.6in]{geometry} % Document margins
\usepackage{url}

\name{Fedor Shmarov} % Your name

\address{+447549260447 \\ shmarovfedor@mail.ru \\ shmarov.com} % Your phone number and email
\address{195 Melbourne Court, Newcastle upon Tyne, NE1 2AT, UK} % Your secondary addess (optional)

\begin{document}

%----------------------------------------------------------------------------------------
%	WORK EXPERIENCE SECTION
%----------------------------------------------------------------------------------------

\begin{rSection}{Personal Profile}

I am an enthusiastic and motivated researcher with a strong background in formal reasoning
and model checking, currently working on a Rosetrees Fund sponsored project
 where I apply formal methods and machine learning techniques
to designing personalised ultraviolet (UVB) photo-therapies for treating psoriasis.
% using collected clinical data.
I am particularly interested into conducting a research into 
combining machine learning and formal verification
({\em e.g.}, ensuring formal correctness of the machine learning algorithms,
or using machine learning to speed up formal verification).

\end{rSection}


%----------------------------------------------------------------------------------------
%	EDUCATION SECTION
%----------------------------------------------------------------------------------------

\begin{rSection}{Education}


{\bf Newcastle University, UK} \hfill {\em 2013 - 2018} \\ 
Ph.D. in Computing Science  \\
{\em Thesis Title}: Probabilistic Bounded Reachability for Stochastic Hybrid Systems \\ 
{\em Supervisor}: Dr Paolo Zuliani
\smallskip 

{\bf Newcastle University, UK} \hfill {\em 2012 - 2013} \\ 
M.Sc. (with Distinction) in Advanced Computer Science
\smallskip 

{\bf Tambov State Technical University, Russia} \hfill {\em 2007 - 2011} \\ 
B.Sc. (with Honours) in Information Science and Computer Technology
\smallskip 

\end{rSection}


%----------------------------------------------------------------------------------------
%	WORK EXPERIENCE SECTION
%----------------------------------------------------------------------------------------

\begin{rSection}{Work Experience}


{\bf Newcastle University, UK} \hfill {\em 2017 - present} \\ 
Research Associate in School of Computing \\

{\bf Projects}: \\
{\em Project Title}: Personalised ultraviolet B (UVB) treatment of psoriasis through biomarker integration with computational modelling of psoriatic plaque resolution \\ 
{\em Principal Investigators}: Dr Paolo Zuliani and Prof Nick Reynolds \\
{\em Sponsors}: Rosetrees Trust

\end{rSection}



%----------------------------------------------------------------------------------------
%	Research EXPERIENCE SECTION
%----------------------------------------------------------------------------------------

% \begin{rSection}{Research Experience}

% From October 2017 I have been working on the project ...
% My main contributions:
% \begin{itemize}

% 	\item ...


% \end{itemize}

% From September 2013 to January 2017 I was involved in the project
% {\em "Verification of Complex Cyber-Physical Systems"}
% sponsored by US Office of Naval Research. My main contributions:
% \begin{itemize}

% 	\item Developed novel methods combining formal verification techniques
% 	and statistical model checking for probabilistic reachability analysis in
% 	stochastic parametric hybrid systems. 
% 	\item Developed a technique based on formal reasoning for parameter set synthesis in 
% 	parametric hybrid systems.
% 	\item Developed \texttt{ProbReach} -- tool for probabilistic reachability 
% 	analysis in stochastic parametric hybrid systems. It is implemented in C++ and 
% 	available at \url{https://github.com/dreal/probreach}.
% 	\item Participated in the development of \texttt{BioPSy} -- tool
% 	for parameter set synthesis in biological models. 
% 	In this work I implemented the back-end algorithm and several visualisation
% 	features. \texttt{BioPSy} implemented in C++ (back-end) and Java (front-end). 
% 	The tool is available at \url{https://github.com/dreal/biology}.

% \end{itemize}


% \end{rSection}
%------------------------------------------------

\begin{rSection}{Teaching Experience}

\begin{itemize}
	\item As Research Associate I have co-supervised final projects for
	several undergraduate and master students.

	\item As PhD student I worked as a demonstrator for several
	school modules for undergraduate and master students.
	% \begin{itemize}
	% 	\item Programming in Java
	% 	\item System Validation
	% 	\item Understanding Concurrency
	% 	\item Mathematics
	% \end{itemize}
\end{itemize}

\end{rSection}


\begin{rSection}{Technical Skills}

\begin{itemize}

	\item Programming Languages: C/C++, Java, Python, MATLAB, R
	\item Operating Systems: Linux, Windows, Mac OS

\end{itemize}

\end{rSection}

\begin{rSection}{Personal Skills}

\begin{itemize}

	\item Languages: English (fluent), Russian (native)
	\item Experienced public speaker and presenter

\end{itemize}

\end{rSection}



%----------------------------------------------------------------------------------------
%	TECHNICAL STRENGTHS SECTION
%----------------------------------------------------------------------------------------

\begin{rSection}{Publications}

\begin{itemize}

	\item {\bf F. Shmarov}, N. Paoletti, E. Bartocci, S. Lin, S. Smolka and P. Zuliani. 
	``SMT-based Synthesis of Safe and Robust PID Controllers for Stochastic Hybrid Systems''.
	{\em Proceedings of the 13th International Haifa Verification Conference (HVC 2017)}, pp. 131-146.

	\item {\bf F. Shmarov}, P. Zuliani, ``Probabilistic Hybrid Systems Verification via 
	SMT and Monte Carlo Techniques'' in {\em HVC}. LNCS, vol. 10028, 2016,
	pp. 152--168.

	\item {\bf F. Shmarov} and P. Zuliani, ``SMT-based Reasoning for Uncertain Hybrid Domains,'' 
	in {\em AAAI-16 Workhop on Planning for Hybrid Systems, 
	30th AAAI Conference on Artificial Intelligence}, 2016, pp. 624--630.

	\item C. Madsen, {\bf F. Shmarov}, and P. Zuliani, ``BioPSy: an SMT-based Tool for 
	Guaranteed Parameter Set Synthesis of Biological Models,'' in {\em CMSB},
	ser. LNCS, vol. 9308, 2015, pp. 182--194.

	\item {\bf F. Shmarov} and P. Zuliani, ``ProbReach: a Tool for 
	Guaranteed Reachability Analysis of stochastic parametric hybrid systems,''
	in {\em Symbolic and Numerical Methods for Reachability Analysis, 1st International Workshop, SNR 2015}, 
	ser. EPiC Series in Computing, S. Bogomolov and A. Tiwari, Eds., vol. 37, 2015, pp. 40--48.

	\item {\bf F. Shmarov} and P. Zuliani, ``ProbReach: Verified Probabilistic 
	Delta-Reachability for stochastic parametric hybrid systems,''
	in {\em HSCC}. ACM, 2015, pp. 134--139.

\end{itemize}

\end{rSection}



\begin{rSection}{Workshop Presentations}

\begin{itemize}

	\item {\bf F. Shmarov}. "Probabilistic Bounded Reachability for Stochastic Hybrid Systems". 
	{\em Third Workshop on Design and Analysis of Robust Systems (DARS), 2018}.

	\item {\bf F. Shmarov} and P. Zuliani. "ProbReach: Probabilistic Bounded 
	Reachability for Uncertain Hybrid Systems". {\em International Workshop on
	Formal Methods for Rigorous Systems Engineering of Cyber-Physical Systems (RiSE4CPS), 2017}.

	\item {\bf F. Shmarov}. "Stochastic Hybrid Systems: Modelling Cancer and Psoriasis". 
	{\em International Workshop on Automated Reasoning for Systems Biology and Medicine (ARSBM), 2016.}

\end{itemize}	

\end{rSection}



\begin{rSection}{Awards}

\begin{itemize}
	\item In 2013 I received Philip Merlin prize from the School of Computing Science 
	for the Best MSc Dissertation.
\end{itemize}

\end{rSection}



\begin{rSection}{Referees}

\end{rSection}


% \begin{rSection}{Technical Skills}

% \begin{itemize}
% 	\item Java, C++
% 	\item Formal verification
% 	\item Statistical Model Checking
% \end{itemize}	

% \end{rSection}

% \begin{rSection}{Personal Skills}

% \begin{itemize}
% 	\item Learning capability and self-education,
% 	\item Managerial and teamwork abilities,
% 	\item Single-mindedness, responsibility,
% 	\item Desire for professional improvement,
% 	\item High capacity for work, diligence.
% \end{itemize}

% \end{rSection}

% % Languages
% \begin{rSection}{Language Skills}
% \begin{itemize}
% 	\item Russian (native),
% 	\item English (fluent).
% \end{itemize}
% \end{rSection}


%----------------------------------------------------------------------------------------
%	EXAMPLE SECTION
%----------------------------------------------------------------------------------------

%\begin{rSection}{Section Name}

%Section content\ldots

%\end{rSection}

%----------------------------------------------------------------------------------------

\end{document}
