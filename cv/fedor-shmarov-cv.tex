%%%%%%%%%%%%%%%%%%%%%%%%%%%%%%%%%%%%%%%%%
% Medium Length Professional CV
% LaTeX Template
% Version 2.0 (8/5/13)
%
% This template has been downloaded from:
% http://www.LaTeXTemplates.com
%
% Original author:
% Trey Hunner (http://www.treyhunner.com/)
%
% Important note:
% This template requires the resume.cls file to be in the same directory as the
% .tex file. The resume.cls file provides the resume style used for structuring the
% document.
%
%%%%%%%%%%%%%%%%%%%%%%%%%%%%%%%%%%%%%%%%%

%----------------------------------------------------------------------------------------
%	PACKAGES AND OTHER DOCUMENT CONFIGURATIONS
%----------------------------------------------------------------------------------------

\documentclass{resume} % Use the custom resume.cls style

% \makeatletter
% \newlength{\bibhang}
% \setlength{\bibhang}{1em} %1em}
% \newlength{\bibsep}
%  {\@listi \global\bibsep\itemsep \global\advance\bibsep by\parsep}
% \newenvironment{bibsection}%
%         {\begin{enumerate}{}{%
% %        {\begin{list}{}{%
%        \setlength{\leftmargin}{\bibhang}%
%        \setlength{\itemindent}{-\leftmargin}%
%        \setlength{\itemsep}{\bibsep}%
%        \setlength{\parsep}{\z@}%
%         \setlength{\partopsep}{0pt}%
%         \setlength{\topsep}{0pt}}}
%         {\end{enumerate}\vspace{-.6\baselineskip}}
% %        {\end{list}\vspace{-.6\baselineskip}}
% \makeatother

\usepackage[left=0.75in,top=0.6in,right=0.75in,bottom=0.6in]{geometry} % Document margins
\usepackage{xcolor}
\usepackage{enumitem}
\usepackage{url}
\usepackage{hyperref}


\hypersetup{colorlinks,breaklinks,linkcolor=blue,urlcolor=blue,anchorcolor=blue,citecolor=blue}

\name{Fedor Shmarov} % Your name

\address{195 Melbourne Court, Howard Street, Newcastle upon Tyne, NE1 2AT, UK} % Your secondary addess (optional)
\address{+447549260447 \\ mailto: \texttt{shmarovfedor@mail.ru} \\ \url{www.shmarov.com}} % Your phone number and email

% \address{1 Science Square, Newcastle upon Tyne, NE4 5TG, UK} % Your secondary addess (optional)
% \address{+447549260447 \\ \texttt{fedor.shmarov@newcastle.ac.uk} \\ \url{www.shmarov.com}} % Your phone number and email
% \address{School of Computing, Newcastle University, Urban Sciences Building,}

\begin{document}

%----------------------------------------------------------------------------------------
%	SUMMARY
%----------------------------------------------------------------------------------------

\begin{rSection}{Personal Profile}

I am an enthusiastic and motivated researcher with a strong background in formal verification
and model checking. Currently I am working on a project sponsored by the Rosetrees Trust 
where I perform mathematical modelling of ultraviolet phototherapy for treating psoriasis and 
apply machine learning techniques to clinical data for designing personalised treatments.

%I am particularly interested into conducting a research into 
%combining machine learning and formal verification
%({\em e.g.}, ensuring formal correctness of the machine learning algorithms,
%or using machine learning to speed up formal verification).

\end{rSection}

%----------------------------------------------------------------------------------------
%       WORK EXPERIENCE SECTION
%----------------------------------------------------------------------------------------

\begin{rSection}{Work Experience}


{\bf Newcastle University, UK} \hfill {\em October 2017 - present} \\
Research Associate in the School of Computing \\

{\em Project Title}: Personalised ultraviolet B (UVB) treatment of psoriasis through biomarker 
integration with computational modelling of psoriatic plaque resolution \\
{\em Principal Investigators}: Dr Paolo Zuliani and Prof Nick Reynolds \\
{\em Sponsor}: Rosetrees Trust %\\
%{\em Achievements}: 
%\begin{itemize}	
%	\item I performed latent class analysis of clinical data of patients with psoriasis 
%		where the individuals were stratified into distinct groups based on the early response to the therapy.
%	\item I applied machine learning for developing a classifier for identifying
%		patients from the obtained classes.
%	\item I developed a personalised mathematical model of ultraviolet phototherapy for treating psoriasis.
%\end{itemize}
\end{rSection}



%----------------------------------------------------------------------------------------
%	EDUCATION SECTION
%----------------------------------------------------------------------------------------

\begin{rSection}{Education}


{\bf Newcastle University, UK} \hfill {\em September 2013 - January 2018} \\ 
Ph.D. in Computing Science  \\
{\em Thesis Title}: \href{http://hdl.handle.net/10443/4046}{Probabilistic Bounded Reachability for Stochastic Hybrid Systems} \\ 
{\em Supervisor}: Dr Paolo Zuliani \\
{\em Summary}: I developed novel methods combining formal verification techniques
and statistical model checking for probabilistic reachability analysis
% and parameter set synthesis in 
of stochastic parametric hybrid systems. 
I implemented these methods in the tool 
called ProbReach (\url{https://github.com/dreal/probreach}).

\smallskip 

{\bf Newcastle University, UK} \hfill {\em September 2012 - August 2013} \\ 
M.Sc. (with Distinction) in Advanced Computer Science \\
{\em Average grade}: 87.28 out of 100 \\
{\bf {\em Awards}}: The Philip Merlin prize from the School of Computing Science 
for best dissertation by an MSc taught student 2012-2013.
\smallskip 

{\bf Tambov State Technical University, Russia} \hfill {\em September 2007 - July 2011} \\ 
B.Sc. (with Honours) in Information Science and Computer Technology \\
{\em Average grade}: 5.0 out of 5.0
\smallskip 

\end{rSection}



%----------------------------------------------------------------------------------------
%	Research EXPERIENCE SECTION
%----------------------------------------------------------------------------------------

% \begin{rSection}{Research Experience}

% From October 2017 I have been working on the project ...
% My main contributions:
% \begin{itemize}

% 	\item ...


% \end{itemize}

% From September 2013 to January 2017 I was involved in the project
% {\em "Verification of Complex Cyber-Physical Systems"}
% sponsored by US Office of Naval Research. My main contributions:
% \begin{itemize}

% 	\item Developed novel methods combining formal verification techniques
% 	and statistical model checking for probabilistic reachability analysis in
% 	stochastic parametric hybrid systems. 
% 	\item Developed a technique based on formal reasoning for parameter set synthesis in 
% 	parametric hybrid systems.
% 	\item Developed \texttt{ProbReach} -- tool for probabilistic reachability 
% 	analysis in stochastic parametric hybrid systems. It is implemented in C++ and 
% 	available at \url{https://github.com/dreal/probreach}.
% 	\item Participated in the development of \texttt{BioPSy} -- tool
% 	for parameter set synthesis in biological models. 
% 	In this work I implemented the back-end algorithm and several visualisation
% 	features. \texttt{BioPSy} implemented in C++ (back-end) and Java (front-end). 
% 	The tool is available at \url{https://github.com/dreal/biology}.

% \end{itemize}


% \end{rSection}
%------------------------------------------------

%----------------------------------------------------------------------------------------
%	TECHNICAL STRENGTHS SECTION
%----------------------------------------------------------------------------------------

\begin{rSection}{Publications}
{\bf Work in progress}
\begin{enumerate}
	\item N. Watson, N. Wilson, {\bf F. Shmarov}, P. Zuliani, G. Smith, N.J. Reynolds, and S.C. Weatherhead. 
	``Towards predicting response to phototherapy in psoriasis using clinical and serum biomarkers 
	with machine learning techniques''.

	\item {\bf F. Shmarov}, P. Zuliani, G. Smith and N.J. Reynolds. 
	``Modelling of Psoriasis Area Severity Index through ODE remodelling of UVB therapy for psoriasis''.

	\item {\bf F. Shmarov}, P. Zuliani.
	``Probabilistic Bounded Reachability for Stochastic Parametric Hybrid Systems''.
\end{enumerate}


{\bf Under review}
\begin{enumerate}[resume]
	\item {\bf F. Shmarov}, S. Soudjani, N. Paoletti, E. Bartocci, S. Lin, S.A. Smolka, and P. Zuliani.
	``Automated Synthesis of Safe Digital Controllers for Sampled-Data Stochastic Nonlinear Systems''.
	{\em IEEE Access}.

	\item M. Vasileva, {\bf F. Shmarov}, and P. Zuliani.
	``Probabilistic Reachability for Uncertain Stochastic Hybrid Systems via Gaussian Processes''.
	{\em 18th ACM-IEEE International Conference on Formal Methods and Models for System Design}.
\end{enumerate}


{\bf Conferences}
\begin{enumerate}[resume]
	\item A. Abate, H. Blom, N. Cauchi1, J. Delicaris, S. Haesaert, A. Hartmanns, M. Khaled, A. Lavaei,
	C. Pilch, A. Remke, S. Schupp, {\bf F. Shmarov}, S. Soudjani, A. Thorpe, A.P. Vinod, B. Wooding, 
	and P. Zuliani. ``ARCH-COMP20 Category Report: Stochastic Models''. 
	{\em To appear in the proceedings of the 7th International Workshop on Applied 
	Verification of Continuous and Hybrid Systems (ARCH20).}

	\item {\bf F. Shmarov}, N. Paoletti, E. Bartocci, S. Lin, S. Smolka, and P. Zuliani. 
	``SMT-based Synthesis of Safe and Robust PID Controllers for Stochastic Hybrid Systems''.
	{\em Proceedings of the 13th International Haifa Verification Conference (HVC 2017)}. 2017, pp. 131--146.

	\item {\bf F. Shmarov}, and P. Zuliani. ``Probabilistic Hybrid Systems Verification via 
	SMT and Monte Carlo Techniques'' in {\em HVC}. LNCS, vol. 10028, 2016, pp. 152--168. 
	{\bf {\em (Presented by me at the conference.)}}

	\item {\bf F. Shmarov}, and P. Zuliani. ``SMT-based Reasoning for Uncertain Hybrid Domains,'' 
	in {\em AAAI-16 Workhop on Planning for Hybrid Systems, 
	30th AAAI Conference on Artificial Intelligence}, 2016, pp. 624--630.
	{\bf {\em (Presented by me at the workshop.)}}

	\item C. Madsen, {\bf F. Shmarov}, and P. Zuliani. ``BioPSy: an SMT-based Tool for 
	Guaranteed Parameter Set Synthesis of Biological Models,'' in {\em CMSB},
	ser. LNCS, vol. 9308, 2015, pp. 182--194.

	\item {\bf F. Shmarov}, and P. Zuliani. ``ProbReach: a Tool for 
	Guaranteed Reachability Analysis of stochastic parametric hybrid systems,''
	in {\em Symbolic and Numerical Methods for Reachability Analysis, 1st International Workshop, SNR 2015}, 
	ser. EPiC Series in Computing, S. Bogomolov and A. Tiwari, Eds., vol. 37, 2015, pp. 40--48.
	{\bf {\em (Presented by me at the workshop.)}}

	\item {\bf F. Shmarov}, and P. Zuliani. ``ProbReach: Verified Probabilistic 
	Delta-Reachability for stochastic parametric hybrid systems,''
	in {\em HSCC}. ACM, 2015, pp. 134--139.
\end{enumerate}
\end{rSection}


\begin{rSection}{Workshop Presentations}
\begin{itemize}

	\item {\bf F. Shmarov}, P. Zuliani, G. Smith, N. Reynolds. "A Mechanistic Model of Psoriatic
	Epidermis and Psoriasis Therapies". {\em Poster session at MRC PSORT consortium showcase, 2019.}

	\item {\bf F. Shmarov}. "Probabilistic Bounded Reachability for Stochastic Hybrid Systems". 
	{\em Third Workshop on Design and Analysis of Robust Systems (DARS), 2018}.

	\item {\bf F. Shmarov} and P. Zuliani. "ProbReach: Probabilistic Bounded 
	Reachability for Uncertain Hybrid Systems". {\em International Workshop on
	Formal Methods for Rigorous Systems Engineering of Cyber-Physical Systems (RiSE4CPS), 2017}.

	\item {\bf F. Shmarov}. "Stochastic Hybrid Systems: Modelling Cancer and Psoriasis". 
	{\em International Workshop on Automated Reasoning for Systems Biology and Medicine (ARSBM), 2016.}

\end{itemize}	
\end{rSection}

\pagebreak

\begin{rSection}{Technical Skills}
\begin{itemize}
	\item Programming Languages: C/C++, Java, Python, MATLAB, R, SQL
	\item Version control: Git, SVN
	\item Other: HPC, AWS
	\item Operating Systems: Linux, Windows, OS X
\end{itemize}
\end{rSection}


\begin{rSection}{Software Tools}
\begin{itemize}
	\item ProbReach (\url{https://github.com/dreal/probreach}) - tool for formal and statistical
	verification of stochastic parametric hybrid systems.
	\begin{itemize}
		\item Role: main developer
		\item Language: C/C++
	\end{itemize}
	\item BioPSy (\url{https://github.com/dreal/biology}) - tool for parameter set synthesis
	in dynamical systems.
	\begin{itemize}
		\item Role: co-developer
		\item Language: C/C++, Java
	\end{itemize}
	\item System for area planning (\url{https://github.com/shmarovfedor/area-planning}) - 
	tool for maximising the construction profit for some predefined region (MSc project).
	\begin{itemize}
		\item Role: sole developer
		\item Language: Java
	\end{itemize}
\end{itemize}
\end{rSection}


\begin{rSection}{Teaching Experience}
\begin{itemize}
	\item As Research Associate I have co-supervised 1 BSc student (who implemented an extension
	to my tool ProbReach as a part of the final project)
	and 6 Bioinformatics MSc students.

	\item As PhD student I demonstrated and marked assignments for 6 different
	 BSc and MSc modules:
	\begin{itemize}
		\item CSC1021 ``Programming I'',
		\item CSC8105 ``System Validation'',
		\item CSC3324 ``Understanding Concurrency'',
		\item CSC1025 ``Mathematics for Computer Science'',
		\item CSC8317 ``Introductory Programming for Biologists'',
		\item CSC1024 ``Computer Architecture''.
	\end{itemize}
\end{itemize}
\end{rSection}

% \pagebreak

\begin{rSection}{Personal Skills}
\begin{itemize}
	\item Experienced public speaker and presenter
	\item Ability to work in a team
	\item Efficient time management
	\item Languages: English (fluent), Russian (native)
\end{itemize}
\end{rSection}





% \begin{rSection}{Awards}
% \begin{itemize}
% 	\item In 2013 I received the Philip Merlin prize from the School of Computing Science 
% 	for the Best MSc Dissertation.
% \end{itemize}
% \end{rSection}



%\begin{rSection}{References}

%\end{rSection}


% \begin{rSection}{Technical Skills}

% \begin{itemize}
% 	\item Java, C++
% 	\item Formal verification
% 	\item Statistical Model Checking
% \end{itemize}	

% \end{rSection}

% \begin{rSection}{Personal Skills}

% \begin{itemize}
% 	\item Learning capability and self-education,
% 	\item Managerial and teamwork abilities,
% 	\item Single-mindedness, responsibility,
% 	\item Desire for professional improvement,
% 	\item High capacity for work, diligence.
% \end{itemize}

% \end{rSection}

% % Languages
% \begin{rSection}{Language Skills}
% \begin{itemize}
% 	\item Russian (native),
% 	\item English (fluent).
% \end{itemize}
% \end{rSection}


%----------------------------------------------------------------------------------------
%	EXAMPLE SECTION
%----------------------------------------------------------------------------------------

%\begin{rSection}{Section Name}

%Section content\ldots

%\end{rSection}

%----------------------------------------------------------------------------------------

\end{document}
